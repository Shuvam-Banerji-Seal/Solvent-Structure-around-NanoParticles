\documentclass[aspectratio=169]{beamer}

\usetheme{Madrid}
\usecolortheme{default}

\title{Solvation Structure Around Nanoparticles: \\ Hydrophobic vs. Hydrophilic Behavior}
\author{Shuvam Banerji Seal}
\institute{}
\date{\today}

\begin{document}

\begin{frame}
\titlepage
\end{frame}

\section{Motivation: Important Question}

\begin{frame}{Motivation: The Important Question}
\begin{itemize}
\item How does the solvation structure change around nanoparticles as we transition from hydrophobic to hydrophilic behavior?
\item Nanoparticles interact with water differently based on their surface chemistry
\item Understanding this can help design better nanomaterials for applications in drug delivery, catalysis, and environmental remediation
\end{itemize}

\begin{block}{Key Research Question}
How does varying the solute-water interaction strength (hydrophobicity $\leftrightarrow$ hydrophilicity) affect:
\begin{itemize}
\item Radial Distribution Functions (RDFs) of water around the nanoparticle
\item Hydration shell structure and dynamics
\item Coordination numbers and residence times
\end{itemize}
\end{block}
\end{frame}

\section{Literature Review}

\begin{frame}{Literature Context}
\begin{itemize}
\item Molecular dynamics (MD) simulations widely used to study solvation phenomena
\item Lennard-Jones (LJ) potentials commonly employed to model nanoparticle-water interactions
\item Epsilon ($\epsilon$) parameter in LJ potential controls attraction strength
\item Previous studies show hydrophobic solutes have depleted hydration shells, while hydrophilic ones have enhanced water ordering
\end{itemize}

\begin{block}{Relevant Literature}
\begin{itemize}
\item TIP4P/2005 water model for accurate structural properties
\item LAMMPS for flexible force field implementations
\item RDF analysis for structural characterization
\item Hydration number calculations via integration of first solvation shell
\end{itemize}
\end{block}
\end{frame}

\begin{frame}{Literature in Project Context}
\begin{itemize}
\item Building on classical solvation studies using coarse-grained models
\item Focus on systematic parameter sweeps rather than specific nanoparticle types
\item Emphasis on computational efficiency for parameter exploration
\item Integration of analysis tools for automated RDF and hydration number computation
\end{itemize}

\begin{block}{Gap in Literature}
While individual studies exist for specific systems, there is a need for:
\begin{itemize}
\item Systematic exploration of hydrophobicity-hydrophilicity continuum
\item Automated pipelines for large-scale parameter sweeps
\item Clear comparison of structural metrics across interaction strengths
\end{itemize}
\end{block}
\end{frame}

\section{Method/Pipeline}

\begin{frame}{Planned Methodology}
\begin{itemize}
\item Use LAMMPS molecular dynamics simulator for production runs
\item Model nanoparticle as single Lennard-Jones sphere
\item TIP4P/2005 water model for accurate solvation structure
\item Systematically vary $\epsilon$ parameter in solute-water LJ interaction
\item Perform multiple independent replicas for statistical validity
\end{itemize}

\begin{block}{Simulation Protocol}
\begin{enumerate}
\item Energy minimization
\item NVT equilibration (thermalization)
\item NPT equilibration (density stabilization)
\item Production run (data collection)
\end{enumerate}
\end{block}
\end{frame}

\begin{frame}{Pipeline Implementation}
\begin{itemize}
\item Configuration-driven approach using YAML files
\item Automated input file generation for parameter sweeps
\item PACKMOL for initial system setup
\item GPU-accelerated LAMMPS runs (A100/A40/A600 available)
\item Post-processing for RDF computation and hydration number analysis
\end{itemize}

\begin{block}{Key Tools}
\begin{itemize}
\item LAMMPS: MD simulation engine
\item PACKMOL: Initial molecular packing
\item Python ecosystem: Automation and analysis
\item VMD/TopoTools: System preparation
\end{itemize}
\end{block}
\end{frame}

\begin{frame}{Parameter Sweep Design}
\begin{itemize}
\item Epsilon values: [0.02, 0.05, 0.1, 0.2, 0.5, 1.0] kcal/mol
\item Box size: 60 Å cubic box
\item 3 independent replicas per epsilon value
\item Production time: 50 ns per simulation
\item RDF computed every 100 steps with 200 bins
\end{itemize}

\begin{block}{Analysis Metrics}
\begin{itemize}
\item g(r): Radial distribution function
\item N$_\text{coord}$: Coordination number (hydration number)
\item Statistical error bars from replicas
\end{itemize}
\end{block}
\end{frame}

\section{Project Goals}

\begin{frame}{Clear Project Goals}
\begin{itemize}
\item Develop a robust computational pipeline for solvation structure studies
\item Characterize the transition from hydrophobic to hydrophilic solvation
\item Establish quantitative relationships between interaction strength and structural properties
\item Validate methodology against literature benchmarks
\end{itemize}

\begin{block}{Specific Objectives}
\begin{enumerate}
\item Implement automated parameter sweep workflow
\item Generate high-quality RDF data across epsilon range
\item Compute hydration numbers and analyze trends
\item Document pipeline for reproducibility
\end{enumerate}
\end{block}
\end{frame}

\begin{frame}{Expected Outcomes}
\begin{itemize}
\item Clear identification of hydrophobic ($\epsilon \to 0$) vs. hydrophilic ($\epsilon \to 1$) regimes
\item Quantitative metrics for solvation shell properties
\item Insights into nanoparticle design for specific applications
\item Reusable computational framework for future studies
\end{itemize}

\begin{block}{Success Criteria}
\begin{itemize}
\item Converged RDFs showing systematic changes with $\epsilon$
\item Statistically significant hydration number trends
\item Pipeline capable of handling 18+ simulations efficiently
\end{itemize}
\end{block}
\end{frame}

\begin{frame}{Timeline and Milestones}
\begin{itemize}
\item Month 1: Pipeline development and testing
\item Month 2: Parameter sweep execution
\item Month 3: Data analysis and validation
\item Month 4: Results interpretation and documentation
\end{itemize}

\begin{block}{Deliverables}
\begin{itemize}
\item Complete simulation dataset
\item Analysis scripts and visualization
\item Publication-ready figures
\item Reproducible workflow documentation
\end{itemize}
\end{block}
\end{frame}

\end{document}